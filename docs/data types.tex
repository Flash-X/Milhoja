\documentclass[letterpaper,12pt]{article}
\usepackage{tabularx} % extra features for tabular environment
\usepackage{amsmath}  % improve math presentation
\usepackage{graphicx} % takes care of graphic including machinery\
\usepackage[margin=1in,letterpaper]{geometry} % decreases margins
\usepackage{cite} % takes care of citations
\usepackage[final]{hyperref} % adds hyper links inside the generated pdf file
\hypersetup{
	colorlinks=true,       % false: boxed links; true: colored links
	linkcolor=blue,        % color of internal links
	citecolor=blue,        % color of links to bibliography
	filecolor=magenta,     % color of file links
	urlcolor=blue         
}

%++++++++++++++++++++++++++++++++++++++++


\begin{document}

\title{Orchestration Runtime Notes}
\author{Tom Klosterman}
\date{\today}
\maketitle

\begin{abstract}

\end{abstract}


\section{Orchestration Namespace}

The Grid and Tile classes, as well as related classes and functions, live inside the Orchestration namespace.


\section{Real Type}

For the sake of flexibility in type-matching, the Orchestration System maintains its own floating point type: orchestration::Real. This is a typedef for either double or float, depending on the macro defined at compilation, REAL\_IS\_DOUBLE or REAL\_IS\_FLOAT.

\section{Vectors}
\subsection{IntVect}

The IntVect class represents NDIM-tuples of integers. Most frequently, they represent vectors in the index-space of the domain. Users are responsible of tracking whether they represent cell-based or node-based indices. They have basic math operators defined, such as component-wise addition, scalar multiplication, etc.

\subsection{IntTriple}

The IntTriple class represents MDIM-tuples of integers. They are used to write dimension-agnostic loops and to create FArray4Ds. They do NOT have math operators defined as they aren't intended to be used as general purpose vectors.



\subsection{RealVect}

The RealVect class represents NDIM-tuples of Reals. They typically represent points of the domain in physical space. They have basic math operators defined in the same way as IntVect.

\section{Arrays}
\subsection*{FArray4D}

The FArray4D class functions primarily as a wrapper to Real pointers that provides Fortran-like access to their data with an overload of \texttt{operator()}. Notably, the access pattern is Fortran-style column major.



\end{document}
