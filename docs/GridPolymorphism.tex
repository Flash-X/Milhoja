\documentclass[letterpaper,12pt]{article}
\usepackage{tabularx} % extra features for tabular environment
\usepackage{amsmath}  % improve math presentation
\usepackage{graphicx} % takes care of graphic including machinery\
\usepackage[margin=1in,letterpaper]{geometry} % decreases margins
\usepackage{cite} % takes care of citations
\usepackage[final]{hyperref} % adds hyper links inside the generated pdf file
\hypersetup{
	colorlinks=true,       % false: boxed links; true: colored links
	linkcolor=blue,        % color of internal links
	citecolor=blue,        % color of links to bibliography
	filecolor=magenta,     % color of file links
	urlcolor=blue         
}

%++++++++++++++++++++++++++++++++++++++++


\begin{document}

\title{Polymorphism in Grid}
\author{Tom Klosterman}
\date{\today}
\maketitle

\begin{abstract}
Section of dev guide explaining how polymorphism was achieved in Grid base classes.
\end{abstract}


\section{Polymorphism}

Base classes in the Grid unit (e.g. Grid, Tile) need to have package-agnostic public interfaces so they can be used in physics code without regard to the AMR package. However, the implementation of these classes is highly dependent on the package, so polymorphism is needed. All of the Grid base classes use the same class design pattern to achieve polymorphism - an abstract base class with virtual methods that are implemented in concrete derived classes (e.g. GridAmrex, TileAmrex). The abstract base classes control the public interface while the derived classes contain private data members and methods as needed for the implementation. However, the way these classes are instantiated and passed through public interfaces varies. Two patterns are explained below, with examples.


\section{Reference-Based Singleton (e.g. Grid)}
The Grid class was designed to be instantiated exactly once, i.e. the Singleton design pattern. Polymorphism is achieved by storing the derived-class object in the static namespace of the Grid::instance() member function. The exact type of derived class is chosen by a preprocessor macro. After the singleton is instantiated, the user can obtain a reference to it by calling Grid::instance(). Note that the dervied class reference decays to a base class reference in the return statement.

\begin{verbatim}
// Access
Grid&   Grid::instance(void) {
    ...
    static GridAmrex gridSingleton;
    return gridSingleton;
}

// Usage

IntVect del = Grid::instance().getDeltas();
//    or
Grid& grid = Grid::instance();
IntVect del = grid.getDeltas();
\end{verbatim}


\section{Smart Pointer Encapsulation (e.g. Tile)}

Unlike Grid, Tile objects are instantiated very frequently throughout physics code. In addition, memory management is very important for Tile objects, since they need to be passed through the Orchestration Runtime data pipeline. To account for these restraints, when derived class Tile objects are created, they should be immediately encapsulated into derived-class smart pointers (default $\texttt{std::unique\_ptr}$) which can then decay into base-class smart pointers to be returned by a public interface. This means a $\texttt{std::unique\_ptr<Tile>}$ object will contain the original derived class object but can be used in a package-agnostic way.

Note that \texttt{buildCurrentTile()}  intentionally returns a $\texttt{std::unique\_ptr}$ so calling code can interpret it as a $\texttt{std::unique\_ptr}$ or $\texttt{std::shared\_ptr}$ as needed.

\begin{verbatim}
// Creation
std::unique_ptr<Tile> TileIterBaseAmrex::buildCurrentTile() override {
    return std::unique_ptr<Tile>{ new TileAmrex(mfi_,lev_) };
}

// Usage
std::unique_ptr<Tile> t = ti->buildCurrentTile();
IntVect coord = t->lo();
\end{verbatim}

TileIter requires the same type of polymorphism as Tile, so the same pattern was used:
\begin{verbatim}
// Creation
std::unique_ptr<TileIter> GridAmrex::buildTileIter(const unsigned int lev) {
    return std::unique_ptr<TileIter>{ new TileIterAmrex(unk_, lev) };
}

// Usage
for (auto ti = grid.makeTileIter(0); ti->isValid(); ti->next() ) {
    std::unique_ptr<Tile> tileDesc = ti->buildCurrentTile();
    ...
}


\end{verbatim}
\end{document}
